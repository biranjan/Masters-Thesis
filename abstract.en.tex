

One of the major barrier for the application of machine learning in both academic and commercial field has been the limited availability of data. Over the past
decades majority of the problem associated with the application of machine
learning has been more or less solved with high speed computing processor and
efficient algorithm. However, data is still illusive as substantial time and money
is spent in collection and preparation of data. This thesis tries to solve
this problem through artificially generating data; process also known as data
augmentation.

In the thesis various augmentation technique namely data-warping,synthetic minority over sampling and generative adversarial nets are explored as a possible
augmentation solution. Models with augmentation and without augmentation are compared and evaluated using harmonic mean of precision and recall.This evaluation metric enables model evaluation based not just on majority class but also based on less frequent minority class. Deep learning model are designed to classify organic compound using the spectral data acquired through Near infrared sensor.

From evaluation of the deep learning model we could observe noticeable improvement in the model with augmentation. Almost all the model performed better over the model without any augmentation. Augmented model did well not only with overall accuracy but also with precision and recall. This thesis tries to tackle contemporary problem using both old and new techniques from very domain specific simple algorithm to the state of the art generative models.