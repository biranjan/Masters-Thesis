\chapter{Conclusions}
\label{chapter:conclusions}

In this thesis, we have brought forward various augmentation frameworks, from simple augmentation using data warping to generative adversarial network. The key findings and the contribution of this paper are discussed below.


\section{Key Findings}

The thesis was set out to improve the classification model using data augmentation. After comparing the models with and without augmentation, we can infer that in the case of NIRs data, augmentation can help improve classification accuracy of model especially in the case of imbalanced data sets. In all the models built for this research, augmentation was used to balance the data set which enhanced the mode performance. In the field of NIR spectroscopy, one of the principal limitations of its application has been the collection of data. As demonstrated in the thesis, augmentation can help mitigate this issue. From the evaluation of various augmentations, we observed that augmentation using the GANs was superior to the augmentation using data warping and SMOTE. 
   
  
\section{Contributions}

As mentioned in the introduction section, there was a primarily two-fold objective of the thesis. One was to provide a theoretical contribution in the field of deep learning and data augmentation, and the other was to propose concrete augmentation framework for the case company to improve the performance of a deep learning model. 

\subsection{Theoretical Contribution}
Data augmentation is a contemporary issue in the field of data science. It has been extensively used in multidimensional data, for example, image detection and computer vision problems. However, data augmentation has not been widely used in one-dimensional data like spectral data or time series data. So through the application in one-dimensional data, this thesis hopes to contribute in growing literature of application of data augmentation.

Similarly, GANs is a concept that was developed only recently, making its application a contemporary research topic in the field of deep learning. By applying this new concept into the business case, we hope to further the understanding of the generative models and their possible implications in businesses.


\subsection{Practical Contribution}

This thesis has applied theoretically grounded methods to tackle a business problem faced by a case company. One of the primary objectives was to produce methods to improve the classification accuracy of existing model. Although the model presented in the thesis requires further evaluation and test, it provides a working framework to improve the model performance using data augmentation. The thesis offers a way to tackle the problem faced by the industry in general. The application if NIRs is restricted due to high data acquisition cost. Data augmentation can save data acquisition cost and extend the application of NIRs in other fields.


Furthermore, all the experiments are done using open source technology and framework. Therefore, findings of the thesis can easily be implemented for product development in the case company. Similarly, other researchers or companies interested in data augmentation can also benefit from this as frameworks for augmentation is laid out in the thesis.


 
